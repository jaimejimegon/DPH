\thispagestyle{empty}
\chapter{Introducción}

Como bien es sabido, el desarrollo de una nueva aeronave partiendo de 0 es un trabajo tremendamente complejo que supondría en la industria unos costes desmesurados. Por ello, en este Trabajo de Fin de Grado se plasmará el desarrollo de un diseño preliminar empleando para ello un análisis de vehículos semejantes ya existentes.\\

La aeronave a desarrollar será un UAV de un MTOW de 450kg de peso, por lo que será fundamental desarrollar una pequeña base de datos de aeronaves similares para poder obtener un primer diseño.
En este capítulo se tratarán, además de los objetivos del trabajo, las bases de la mecánica de vuelo de las aeronaves de ala rotatoria de forma sencilla.

\section{Objetivos del Trabajo}

El objetivo principal de este trabajo es generar un diseño preliminar de una aeronave no tripulada de ala rotatoria de un peso máximo al despegue de 450kg. Este diseño, realizado mediante un estudio de aeronaves similares, será después validado por un análisis de sus actuaciones a partir del equilibrado de la misma.
Es justo aquí donde se verá reflejada la originalidad del trabajo, a la hora de elegir las actuaciones a analizar.\\

La validación se llevará a cabo además de forma paralela a una optimización de características de la aeronave, como pueda ser el alcance o la autonomía, en función de otros parámetros de la misma.

Es importante también definir la misión de la aeronave a diseñar, ya que en función de esta optimizaremos unos parámetros u otros y exigiremos unos mínimos a las actuaciones de la misma.


\section{Uso de las Aeronaves no Tripuladas}
Las aeronaves no tripuladas (\emph{\textbf{UAV} \textbf{U}nmanned \textbf{A}erial \textbf{V}ehicle}) están fuertemente ligadas a la aviación militar, siendo esta industria la responsable principal de su desarrollo a lo largo de su historia.
Aunque anteriormente se dieron casos de \emph{UAVs}, como son los globos con los que el ejercito austríaco bombardeó Venecia en 1849, las primeras aeronaves no tripuladas como las conocemos hoy se desarrollaron durante la Primera Guerra Mundial por parte de los Estados Unidos.\\

A día de hoy, y aunque el desarrollo de la tecnología ha sido gracias a la industria militar, más concretamente a la industria militar de Estados Unidos, su uso se ha extendido más allá de esta. Ya no se trata de instrumentos de guerra ni de un artículo de lujo, sino que existen una gran variedad de aeronaves que cumplen distintas funciones fuera de la aviación militar, sino como parte de la aviación comercial civil (como puedan ser las aeronaves radio control cuyo mando puede ser un \emph{smartphone} personal).\\

Algunos de estos usos son los siguientes:

\begin{itemize}
	\item Fotografía y grabación aérea, tanto profesional como recreativa.
	\item Control de daños en zonas afectadas por desastres.
	\item Transporte de mercancías.
	\item Seguimiento y predicción de fenómenos atmosféricos (tornados, tormentas, etc.)
	\item Control y patrulla de fronteras
	\item Inspecciones en zonas de difícil acceso (o imposible).
	\item Entretenimiento
\end{itemize}

Por otro lado siguen creciendo los usos militares;

\begin{itemize}
	\item Combate aéreo
	\item Supervisión y control
	\item Balizas de objetivos
\end{itemize}

Estos son solo algunos de los usos de los \emph{UAVs}, pero la lista crece continuamente.\\

Pese al fuerte desarrollo civil, el principal gasto mundial en aeronaves no tripuladas viene del sector militar, motivado también por el gasto que conllevan los programas militares y los costes de las aeronaves (en 2011, el coste del programa MQ-1 \emph{Predator} era de 2,38 mil milllones de dólares \cite{Predatorunitbudget}, mientras que el coste de una unidad del mismo se sitúa en 4,03 millones de dólares \cite{Predatorprogrambudget}). Se espera un gasto global de 70 mil millones de dólares en aeronaves no tripuladas para 2020 \cite{Goldman}\\

Sin embargo, el sector cuyo crecimiento se espera sea mayor es el civil. Según datos de BI Intelligence, se espera un crecimiento del 19\% en el mercado civil frente a un 5\% en el militar para el período 20015-2020.
Esto se debe principalmente al incremento en la variedad de operaciones que los \emph{UAVs} son capaces de realizar y a su implementación en las empresas. Debido a este crecimiento, se espera también la creación de 100.000 puestos de trabajo solo en Estados Unidos para 2025 \cite{AUVSI}.\\

\section{Mecánica del Vuelo de un Helicóptero}

%A completar posteriormente junto a lo anterior, de vez en cuando escribe algo mamon.

\section{Descripción del Proyecto}

En este caso se ha decidido desarrollar una aeronave ligera cuyo objetivo será abastecer pequeños despliegues militares en zonas de conflicto desde las bases principales. Los convoyes de transporte por tierra han sido atacados durante los últimos años y por ello las tropas han tenido problemas con el abastecimiento de suministros. La idea es que la aeronave sustituya a estos, de manera que el transporte no solo sea más seguro al ser más difícil de interceptar, sino también más rápido y seguro al no contar con tripulación a bordo. 
Los suministros serán principalmente alimentos, aunque sería posible mandar también armamento y suministros sanitarios en caso de necesidad, aunque siempre en pequeñas cantidades, los despliegues a abastecer no serán muy numerosos, y tampoco deberán estar muy alejados de una base principal.

Esto deja claro las características necesarias para la aeronave; 
\begin{itemize}
	\item Una carga de pago máxima lo más alta posible, de manera que resulte eficiente y no sean necesarios múltiples vuelos diarios para un mismo despliegue salvo casos excepcionales.
	\item Una velocidad de crucero buena, de manera que el tiempo que la aeronave esta en vuelo sea el mínimo posible, para evitar así la intercepción. Además también servirá para transportar suministros médicos en caso de emergencia.
	\item Un alcance de unos 400 Km, se manera que pueda abastecer a distancias de alrededor de 200 Km. Es importante que la aeronave pueda realizar el trayecto de ida y vuelta sin repostar, ya que ello obligaría a disponer de combustible en las zonas a abastecer, lo que no es conveniente para la misión. También hay que tener en cuenta que la mitad del trayecto se realizará con la máxima carga de pago la mayoría de las ocasiones, pero la otra mitad la aeronave no tendrá en la mayoría de los casos carga alguna (será posible enviar de vuelta a las bases principales pequeñas cargas en caso de necesidad).
	\item Un techo de vuelo suficiente para dificultar su intercepción, ya que la aeronave no contará con un blindaje militar para priorizar otras características.
\end{itemize}
Como se observa, apenas se han definido numéricamente estas características, ya que en una primera aproximación no se conoce con exactitud la misión, por lo que todos estos parámetros se intentarán maximizar durante el diseño para obtener unas actuaciones lo mejores posible y que la aeronave pueda ser útil en un rango más amplio de situaciones.


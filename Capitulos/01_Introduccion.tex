\thispagestyle{empty}
\chapter{Introducción}

Como bien es sabido, el desarrollo de una nueva aeronave partiendo de 0 es un trabajo tremendamente complejo que supondría en la industria unos costes desmesurados. Por ello, en este Trabajo de Fin de Grado se plasmará el desarrollo de un diseño preliminar empleando para ello un análisis de vehículos semejantes ya existentes.\\

La aeronave a desarrollar será un UAV de un MTOW de 450kg de peso, por lo que será fundamental desarrollar una pequeña base de datos de aeronaves similares para poder obtener un primer diseño.
En este capítulo se tratarán, además de los objetivos del trabajo, las bases de la mecánica de vuelo de las aeronaves de ala rotatoria de forma sencilla.

\section{Objetivos del Trabajo}

El objetivo principal de este trabajo es generar un diseño preliminar de una aeronave no tripulada de ala rotatoria de un peso máximo al despegue de 450kg. Este diseño, realizado mediante un estudio de aeronaves similares, será después validado por un análisis de sus actuaciones a partir del equilibrado de la misma.
Es justo aquí donde se verá reflejada la originalidad del trabajo, a la hora de elegir las actuaciones a analizar.\\

La validación se llevará a cabo además de forma paralela a una optimización de características de la aeronave, como pueda ser el alcance o la autonomía, en función de otros parámetros de la misma.

Es importante también definir la misión de la aeronave a diseñar, ya que en función de esta optimizaremos unos parámetros u otros y exigiremos unos mínimos a las actuaciones de la misma.


\section{Uso de las Aeronaves no Tripuladas}
Las aeronaves no tripuladas (\emph{\textbf{UAV} \textbf{U}nmanned \textbf{A}erial \textbf{V}ehicle}) están fuertemente ligadas a la aviación militar, siendo esta industria la responsable principal de su desarrollo a lo largo de su historia.
Aunque anteriormente se dieron casos de \emph{UAVs}, como son los globos con los que el ejercito austríaco bombardeó Venecia en 1849, las primeras aeronaves no tripuladas como las conocemos hoy se desarrollaron durante la Primera Guerra Mundial por parte de los Estados Unidos.\\

A día de hoy, y aunque el desarrollo de la tecnología ha sido gracias a la industria militar, más concretamente a la industria militar de Estados Unidos, su uso se ha extendido más allá de esta. Ya no se trata de instrumentos de guerra ni de un artículo de lujo, sino que existen una gran variedad de aeronaves que cumplen distintas funciones fuera de la aviación militar, sino como parte de la aviación comercial civil (como puedan ser las aeronaves radio control cuyo mando puede ser un \emph{smartphone} personal).\\

Algunos de estos usos son los siguientes:

\begin{itemize}
	\item Fotografía y grabación aérea, tanto profesional como recreativa.
	\item Control de daños en zonas afectadas por desastres.
	\item Transporte de mercancías.
	\item Seguimiento y predicción de fenómenos atmosféricos (tornados, tormentas, etc.)
	\item Control y patrulla de fronteras
	\item Inspecciones en zonas de difícil acceso (o imposible).
	\item Entretenimiento
\end{itemize}

Por otro lado siguen creciendo los usos militares;

\begin{itemize}
	\item Combate aéreo
	\item Supervisión y control
	\item Balizas de objetivos
\end{itemize}

Estos son solo algunos de los usos de los \emph{UAVs}, pero la lista crece continuamente.\\

Pese al fuerte desarrollo civil, el principal gasto mundial en aeronaves no tripuladas viene del sector militar, motivado también por el gasto que conllevan los programas militares y los costes de las aeronaves (en 2011, el coste del programa MQ-1 \emph{Predator} era de 2,38 mil milllones de dólares \citep{Predatorunitbudget}, mientras que el coste de una unidad del mismo se sitúa en 4,03 millones de dólares \citep{Predatorprogrambudget}). Se espera un gasto global de 70 mil millones de dólares en aeronaves no tripuladas para 2020 \citep{Goldman}\\

Sin embargo, el sector cuyo crecimiento se espera sea mayor es el civil. Según datos de BI Intelligence, se espera un crecimiento del 19\% en el mercado civil frente a un 5\% en el militar para el período 20015-2020.
Esto se debe principalmente al incremento en la variedad de operaciones que los \emph{UAVs} son capaces de realizar y a su implementación en las empresas. Debido a este crecimiento, se espera también la creación de 100.000 puestos de trabajo solo en Estados Unidos para 2025 \citep{AUVSI}.\\

\section{Mecánica del Vuelo de un Helicóptero}

%A completar posteriormente junto a lo anterior, de vez en cuando escribe algo mamon.
A modo de introducción para el lector, se hará un resumen breve de la mecánica del vuelo de un helicóptero. Si se desea profundizar en el tema o resolver cualquier duda que pudiese surgir durante la lectura, se recomienda acudir a \citet{Cuerva} donde se desarrolla de forma mas exhaustiva y completa.\\

\subsection{Sistemas de referencia}
En la física es muy importante definir correctamente los sistemas de referencia que se emplean en la resolución de cada problema ya que las variables tendrán una forma u otra en función de en cual se definan.
En el caso de un helicóptero, los sitemas de referencia principales son 4, a saber:
\begin{itemize}
	\item Ejes tierra $[O_{T};x_{T},y_{T},z_{T}]$
	\item Ejes cuerpo $[O;x,y,x]$
	\item Ejes árbol $[A;x_{A},y_{A},z_{A}]$
	\item Ejes pala $[E;x_{b},y_{b},z_{b}]$
\end{itemize}

El sistema de ejes tierra es aquel con origen $[O_{T}]$ en la superficie terrestre, $z_{T}$ apuntando en la dirección de la gravedad y $x_{T}$ e $y_{T}$ pertenecientes al plano tangente a la superficie terrestre y formando un triedro a derechas. En este sistema, la posición del helicóptero queda definida por $\textbf{r}^{O}$ , siendo este el vector posición del centro de masas del helicóptero respecto al punto de referencia $O_{T}$.\\

El sistema de ejes cuerpo se define como el triedro a derechas con origen $O$ en el centro de masas de la aeronave y $x$ y $z$ en el plano central, con $x$ dirigido hacia hacia adelante y $z$ hacia abajo en una condición de vuelo normal. La importancia de este sistema radica en los ángulos de Euler, que son los ángulos que forman sus ejes con los ejes tierra, siendo estos:

\begin{itemize}
	\item Guiñada $\Psi$
	\item Cabeceo $\Theta$
	\item Balaceo $\Phi$
\end{itemize}

Estos ángulos se pueden definir como el giro del sistema de ejes cuerpo respecto al de ejes tierra, partiendo de una condición de paralelismo entre ambos, respecto a los ejes $z$, $y$ y $x$ respectivamente. Para facilitar la comprensión del lector se ha añadido el esquema \ref{AEuler} que representan dichos ángulos.\\

\begin{figure}
	\centering
	\includegraphics[width=80mm]{imagenes/AEuler}
	\caption{Representación de los ángulos de Euler, Guiñada, Cabeceo y Balanceo}
	\label{AEuler}
\end{figure}

El sistema de ejes árbol tiene su centro $A$ en la intersección del eje del rotor con el plano del rotor y sus ejes forman un triedro a derechas orientándose $z_{A}$ hacia el lado opuesto al fuselaje y $x$ hacia la parte trasera del helicóptero, perteneciendo al plano del rotor.\\

Por último, el sistema de ejes pala tiene su origen $E$ en la articulación de la pala. La dirección de $x_{b}$ es radial hacia la punta de la pala y la de $y_{b}$ es normal a ella, siendo $z_{b}$ tal que el triedro [$x_{b},y_{b},z_{b}$] sea a derechas.\\

Se puede apreciar que de estos 4 sistemas, los mas relevantes para el estudio serán los de ejes tierra y cuerpo, que sirven para describir la mecánica del vuelo de la aeronave, mientras que los ejes árbol y pala se reservan para el estudio aislado de la física del rotor y de las palas.\\

Una vez definidos los sistemas de referencias se pueden empezar a plantear las ecuaciones.

\subsection{Ecuaciones del movimiento}

Las primeras ecuaciones a considerar  son las de fuerzas y momentos, siendo estas:

\begin{equation}
\mathrm{\textbf{F}}^{ex}+M\mathrm{\textbf{g}}=\frac{\mathrm{d}(M\mathrm{\textbf{V}})}{\mathrm{d}t}=M\left(\frac{\mathrm{d\textbf{V}}}{\mathrm{d}t}\right)_{C}+M(\boldsymbol{\omega}\wedge\mathrm{\textbf{V}})
\end{equation}
\begin{equation}
\mathrm{\textbf{M}}^{ex}=\frac{\mathrm{d}([\mathrm{\textbf{I}}]\boldsymbol{\omega})}{\mathrm{d}t}=[\mathrm{\textbf{I}}]\left(\frac{\mathrm{d}\boldsymbol{\omega}}{\mathrm{d}t}\right)_C+\boldsymbol{\omega}\wedge[\mathrm{\textbf{I}}]\boldsymbol{\omega}
\end{equation}
Donde $\mathrm{\textbf{V}}(t)$ es una velocidad de vuelo cualquiera y $\boldsymbol{\omega}(t)$ la velocidad de giro de los ejes cuerpo respecto a los ejes tierra con con $\mathrm{\textbf{F}}^{ex}$ y $\mathrm{\textbf{M}}^{ex}$ las fuerzas y momentos externos que actúan sobre el centro de gravedad del helicóptero, $[\mathrm{\textbf{I}}]$ el tensor de inercia del vehículo, $M$ su masa y $\mathrm{\textbf{g}}$ el vector aceleración de la gravedad. El subíndice $C$ indica derivadas en ejes cuerpo, tal y como aparece en \citet{Cuerva}.

 en equilibrio, es decir, aquel en el que la resultante de fuerzas y momentos externos al mismo es nula. Esto simplifica enormemente los cálculos
\section[\textbf{HE}licopter and \textbf{RO}tor \textbf{E}quilibrium and \textbf{S}tability toolbox]{HEROES}

%Por aqui explicar que no se harán cálculos directamente, sino que se usará la herramienta. Descripción, sistema de "cajas", incognitas de los problemas, soluciones...
Para el cálculo de las actuaciones de la aeronave se empleará \emph{HEROES}.
\emph{HEROES} es una herramienta de MATLAB desarrollada en conjunto por profesores y alumnos de la Escuela Técnica Superior de Ingeniería Aeronáutica y del Espacio de la Universidad Politécnica de Madrid, principalmente del departamento de Aeronaves y Vehículos Espaciales.\\
Esta herramienta permite un cálculo rápido del equilibrado de un helicóptero (entre otras muchas funciones). Es muy importante saber distinguir en esta herramienta cuáles son las variables de entrada del problema y las de salida. Esto que en primera instancia puede parecer sencillo, requiere un nivel de comprensión alto de la mecánica del vuelo de un helicóptero.\\
Para el caso del equilibrado, las variables de entrada principales serán el modelo de atmósfera y la altura, el modelo del vehículo y las condiciones de vuelo del mismo. \emph{HEROES}, con los datos aportados, generará un modelo adimensional y resolverá las ecuaciones del equilibrado para después dimensionalizar los resultados. Estos resultados son las variables de salida, y entre muchas otras se encuentran las potencias de los rotores, las fuerzas y momentos aerodinámicos y los controles del helicóptero.


\section{Descripción del Proyecto}



Esto deja claro las características necesarias para la aeronave; 
\begin{itemize}
	\item Una carga de pago máxima lo más alta posible, de manera que resulte eficiente y no sean necesarios múltiples vuelos diarios para un mismo despliegue salvo casos excepcionales.
	\item Una velocidad de crucero buena, de manera que el tiempo que la aeronave esta en vuelo sea el mínimo posible, para evitar así la intercepción. Además también servirá para transportar suministros médicos en caso de emergencia.
	\item Un alcance de unos 400 Km, se manera que pueda abastecer a distancias de alrededor de 200 Km. Es importante que la aeronave pueda realizar el trayecto de ida y vuelta sin repostar, ya que ello obligaría a disponer de combustible en las zonas a abastecer, lo que no es conveniente para la misión. También hay que tener en cuenta que la mitad del trayecto se realizará con la máxima carga de pago la mayoría de las ocasiones, pero la otra mitad la aeronave no tendrá en la mayoría de los casos carga alguna (será posible enviar de vuelta a las bases principales pequeñas cargas en caso de necesidad).
	\item Un techo de vuelo suficiente para dificultar su intercepción, ya que la aeronave no contará con un blindaje militar para priorizar otras características.
\end{itemize}
Como se observa, apenas se han definido numéricamente estas características, ya que en una primera aproximación no se conoce con exactitud la misión, por lo que todos estos parámetros se intentarán maximizar durante el diseño para obtener unas actuaciones lo mejores posible y que la aeronave pueda ser útil en un rango más amplio de situaciones.


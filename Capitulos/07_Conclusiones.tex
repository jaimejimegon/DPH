\chapter{Resumen, Conclusiones y Vías de Desarrollo}

A lo largo de este trabajo se ha desarrollado el diseño preliminar de un helicóptero no tripulado de vigilancia \emph{DroneHE}, acompañado en todo momento de un análisis preliminar de sus características. Además, se ha analizado la influencia de las condiciones de vuelo y características de diseño como son la rigidez en batimiento y la posición de diversas cargas de pago en sus actuaciones.

Este desarrollo comienza con un estudio y análisis de los vehículos semejantes existentes, para poder obtener unas características básicas que permitan modelar el nuevo diseño. Este modelo se ha generado empleando la herramienta HEROES, usando para ello como base el modelo computacional de un helicóptero tripulado de mayores dimensiones como es el Bölkow Bo 105.

Se ha analizado su comportamiento para 3 tipos de vuelo: vuelo horizontal a nivel del mar, vuelo en espiral ascendente a nivel del mar y vuelo circular a 1000 m de altitud. De los análisis en base a los parámetros de diseño se ha podido deducir que el efecto que produce la integración de la carga no solo depende de la masa de la misma o su tamaño, sino también de la posición que esta ocupa en el fuselaje, siendo además lo más óptimo centrar la carga todo lo posible sobre el centro de masas de la aeronave en vacío. En lo que respecta a la rigidez en batimiento, se ha observado que un aumento de la misma puede mejorar ligeramente las actuaciones, pero a cambio se obtiene un aumento de las cargas sobre las palas, por lo que para poder elegir correctamente un valor de la misma es conveniente realizar un estudio más amplio de la distribución de cargas sobre las palas, su resistencia estructural y el coste económico de dicho cambio, fuera de los límites de este trabajo.

Gracias a los análisis de las condiciones de vuelo se ha podido observar como varía el comportamiento del \emph{DroneHE} según las condiciones de vuelo y obtener unos valores para las actuaciones del mismo para distintos tipos de vuelo, lo que permite comprobar si el diseño cumple con las condiciones de la misión que se le quiere imponer o, en caso contrario, es necesario un cambio de diseño para cumplir con los objetivos.

Cabe destacar que el diseño obtenido, al estar generado mediante un análisis de aeronaves semejantes pero de tamaño muy diferente en su mayoría y empleando las características físicas y aerodinámicas de una aeronave tripulada de un tamaño mucho mayor, es altamente mejorable a través, por ejemplo, de la inclusión de un mayor número de semejantes en el análisis o el uso de las características de otra aeronave más similar a la de diseño como pueda ser el "Cicaré 7B".

Algunas posibles ampliaciones de este trabajo pudieran ser el análisis de estabilidad de la aeronave con las distintas modificaciones, el estudio de la autonomía para peso variable con el tiempo debido al consumo de combustible o el análisis del vehículo para pesos distintos al MTOW suponiendo diversas cargas.

Como vías de desarrollo de este estudio se plantea el el análisis del control del vehículo y su comportamiento con condiciones de viento adversas, de manera que pueda desarrollarse un sistema de control que adapte el mismo en función de las condiciones de viento de manera automática.
